\documentclass{article}
\usepackage[T1]{fontenc} 
\usepackage[pdftex]{graphicx}
\usepackage{fullpage}
\usepackage{listings}
\usepackage{wasysym}
\usepackage{multicol}

\begin{document}


% FIXME:
% 1) The LCD display needs a font.
% 2) The buttons need a font.


\title{GoodWatch User's Guide}
\author{Travis Goodspeed and Friends}
\maketitle


\begin{multicols}{2}
\noindent
Howdy y'all,

The GoodWatch is a clone of the Casio 3208 watch module, replacing
that module in the CA-53W and CA-506 series watches.  It is built
around the CC430F6137 RFSoC, allowing for years of battery life and
programmability in C.  The second model adds a 70cm amateur radio
transceiver.

\section{Usage}

Applets are accessed in a ring, with the Mode button on the right side
of the watch switching between them.  Sometimes the Mode button will
also have a secondary function, in which case you'll press it twice to
move to the next mode.

\subsection*{Clock}

The watch begins in the Clock applet, and resets to that applet after
a few minute from most other applets.

Pressing the Program button switches to a programming mode, where the
hour, minute and date can be changed; the seconds are held at zero
until the minute is finished being set.

Hold $\div$ to show the date in YYYYMMDD format.  Hold 7 for a
self-test, which ought to display {\tt GOODGOOD} but might otherwise
indicate a trouble code.  Hold 4 to show the Git hash of the firmware
revision.

\subsection*{Stopwatch}

The Stopwatch applet allows you to time things.  Press + to start or
stop the timer, and {\tt 0} to reset it.

\subsection*{RPN Calculator}

The calculator operates in Reverse Polish Notation mode.  The current
implementation operates on 32-bit integers, but floating point support
will come in time.  Pressing Mode the first time will reset the stack
to zero; a second press moves to the next applet.

There's little room here for a complete tutorial for RPN, but gist of
it is that you push numbers onto the stack by typing them in, and each
operator will pop the most recent values from the stack, pushing the
result.  Press {\tt =} to push the current number into the stack and
begin another; you can think of this as the Enter or Space key from an
HP calculator.

\subsection*{Hex Viewer}

The hex viewer application display eight nybbles on the screen.  The
left four are the address, while the right four are the little-endian
value at that address.  The 7, 8, 9, and $\div$ buttons increment
nybbles of the address, while the 0, ., =, and + buttons decrease the
address nybbles.

You will find that unmapped regions of memory return {\tt 0x3FFF}, an
unconditional branch to self.  The bootloader at {\tt 0x1000} is
unlocked for your reading pleasure, but a troublesome region at {\tt
  0x1B00} is blacklisted because reads from that region trigger a
reboot.  Flash runs 32kB starting at {\tt 0x8000}, Info Flash is at
{\tt 0x1800}, and RAM is at {\tt 0x1C00}.  The interrupt table is at
the very end of the address space.


\section{Construction}

\section{Installation}

\section{Source Code}



\end{multicols}

\end{document}
